\section{Labeling} \label{sec:meth_Labeling}

For reasons explained in section \ref{sec:bt_ArtificialIntelligence}, supervised learning requires a labeled dataset to train the learning algorithm for classification. For the specific classification problem which this thesis presents, a dataset with labeled eye movement events is required. These labels will constitute the ground truth for model training. Therefore, its quality will directly translate to the quality of the model produced and the accuracy of predictions. Since the quality of a labeled dataset is tied to the labeling method, this is a crucial step in training a classification algorithm. 

As mentioned in the section above, pre-labeling is done while data is recorded. Even so, these labels are likely to be inconsistent with reality, as they make some assumptions on human reaction times that are not necessarily accurate. Therefore, a second step in the data acquisition pipeline is added to ensure the quality of labeled eye movement events. Here, an operator scans the pre-labeled data files output from the previous step. A .csv editor and a simple plotting script implemented in Python simplify the process. For each event that is not pre-labeled as a fixation, the operator then modifies labels sample-by-sample to match reality.