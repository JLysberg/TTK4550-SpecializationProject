\section{Feature Extraction} \label{sec:meth_FeatureGeneration}

% As described in section \ref{sec:hwds_TobiiEyeTracker5}, user blinking is represented in the data by a loss of samples for the duration of the blink. For blinks differing in periods from 10ms to 100ms, this constitutes the loss of 1 to 5 samples, in the general case. Following the periods of lost data, we can observe large amounts of noise, as the oculomotor muscles attempts to readjust the eye to its preceding point of fixation. To avoid data contamination by these events, we also instruct the user to indicate whenever he is blinking. This way, data contaminated by blinking can be removed as a post-processing operation. Since we cannot guarantee that the user fixates on the exact same point preceding and following a blink, we will split the time series on any blink-labelled sample such that the feature generator of section \ref{sec:meth_FeatureGeneration} does not operate on any sampling window that encapsulate contaminated data. 

The final step of the data acquisition pipeline is responsible for preparing the dataset before classification. Here, if blinks are not already labeled from the static environment, they are automatically inferred by the time difference between samples. Such samples are considered data contamination and removed.

Finally, features are appended to the dataset, which the classifier will use to distinguish between events. Since the machine learning model to be detailed in the next section is rather limited in its hypothesis space, it requires more than just coordinate values to do accurate classifications. This alternate representation is acquired by extracting relevant information from the dataset. The choice of features is inspired from \cite{zemblys2018}, which is again inspired by common or state-of-the-art manual classification algorithms. However, since the Tobii ET5 has a significantly lower sampling rate than the eye tracker used in this paper, many features that rely on small sample intervals have been eliminated. The eight features extracted are presented in table \ref{tab:meth_FeatureTable}.

\import{./}{FeatureTable}