\section{Recording and Labeling} \label{sec:disc_RecordingAndLabeling}

% \subsection{Post-labeling}

% \subsection{Dynamic Environment}

The development of recording environments has been an iterative process. It begins at the barest minimum viable product and ends with an entirely plug-and-play robust application. In the end, it was able to handle all aspects of eye-tracking data management, from data recording to heatmap generation and pre-labeling. Since the raw implementation and operation of such environments were described in section \ref{sec:meth_RecordingEnvironments}, this section will focus on describing the datasets and labels they were able to produce, as well as the reasoning behind the most important choices made along its developmental process. 

\subsection{From Static to Dynamic}

To begin with, the goal of the recording environment was simple: To connect to hardware, gather raw eye-tracking data, and output a dataset sufficient for binary classification. Since this requires no more than binary labels, the entire labeling process could essentially be done at runtime by the subject. For this purpose, a simple environment with a static image and an on-screen labeling button would suffice. The thought was that the user could hold a button when fixating on a point on the image and release it when the focus shifted. Additionally, the user could indicate that blinks contaminated the data output by pressing another button to stop recording temporarily.

As the reader might already have reasoned, this labeling scheme is sub-optimal. To obtain a dataset of high quality with no external biases, we ideally want the subject to take no part in the recording process itself and only focus on the task at hand. Furthermore, if we wanted to use the dataset for more than binary classification, the burden of labeling more than two eye movement events during runtime would be too complex for the subject to handle. 

For these reasons, a natural augmentation was that of a dynamic environment, the details of which are explained in section \ref{sec:meth_RecordingEnvironments}. This development removed the subject from the loop entirely, allowing for less biased and more abundant data since they could record for extended periods before getting tired. Another unexpected improvement of the dynamic environment was the effect of discovering and removing more contaminations by blinks. For the static environment, it turned out that even though the subject was instructed to label their blinks, there were inevitably unconscious blinking that went unnoticed in the final dataset. Even when run through post-labeling, a manual agent found these minor data points, distinguishable only by a few milliseconds larger gaps between timestamps of consecutive samples, very hard to notice. 

The raw labeled datasets as output from both the static and the dynamic recording environments are depicted in figure \ref{fig:res_DatasetsPrelabel}.

\newpage
\subsection{The Value of Post-labeling}

Observing the onsets and offsets of movement events in both datasets of figure \ref{fig:res_DatasetsPrelabel}, however, there is still a very poor correlation between labels and the ground truth. For the statically recorded dataset, this is again caused by involving human error in the labeling process. Both because human reaction time is unpredictable and nonzero and because the user has no way of distinguishing differing lengths of movements in their labeling, the resulting labels are lagging and randomly distributed in their quantity. The dynamically recorded dataset is hardly any better, however. It, too, suffers from error caused by human reaction time, though in the opposite direction in time. Another likely cause is that the data stream from Tobii ET5 is also lagging actual events by some milliseconds. Either way, to obtain the dataset we seek for accurate event classification, these errors need to be addressed.

The solution manifests itself as a manual post-labeling step in the data acquisition pipeline. Although sub-optimal regarding the autonomy of the labeling process, some kind of manual intervention is likely unavoidable to obtain a dataset with labels consistent with reality. Still, knowing that the events labeled by the environment are actual and merely shifted a few samples behind the ground truth, post-labeling becomes a straightforward and smooth process. The final dataset, post-labeled and with contaminated blink samples removed, can be seen in figure \ref{fig:res_DatasetFinal}. As mentioned, this particular subset of samples includes those from the right plot of figure \ref{fig:res_DatasetsPrelabel}. From this, one can ascertain the value of post-processing by observing that labels have been both shifted and stretched in time to resemble the ground truth as closely as possible. Additionally, one can see that most data contaminated by subject blinking is removed.

% As was briefly mentioned in section \ref{sec:meth_RecordingEnvironments}, a bias was expected to be introduced in the dataset by human error when the user is prompted to label their own actions by mouse clicks. In figure \ref{fig:res_DatasetsPrelabel}, we attempt to visualise this bias by using the dispersion feature, described in further detail in section \ref{sec:meth_FeatureGeneration}. 

% As we can see, there is a very poor correlation between the peaks of xy-dispertion and each sample's labelled event. This fact reveals the major value of the post-labeling step, employed in dataset represented above. It seems that labeled saccades are lagging the dispertion peaks by a few samples, likely because the user releases the labelling button as he is moving his gaze, and simple human reaction time leads to the observed bias. Additionally, we can see that the sample period of saccade-labelled samples have little correlation with the number of samples with high values of xy-dispersion, nor the length of the saccade. This bias is likely caused by the fact that the user has no way of distinguishing the micro-differences of a 15ms saccade from a 100ms saccade, and consequently label all saccades with the same interval of button release. 
