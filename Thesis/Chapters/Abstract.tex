\chapter*{Abstract} 
\addcontentsline{toc}{chapter}{Abstract}

This project researches the potential of low-cost commercial-grade eye-tracking hardware using traditional and machine learning methods for automatic eye movement classification. Results are motivated by the possibilities of making informative inferences for performance analytics available for everyone.

Esports is a growing industry, and as kids and adults get progressively into competitive gaming, the need for concrete feedback for performance advancements arises. However, most feedback forms come from direct coaching, which is no reasonable alternative for the casual gamer. \cite{may1990}, among others, prove an immense correlation between ocular data and cognitive states. Along with the emergence of low-cost eye-tracking and increasingly advanced machine learning methods, this fact advocates further work towards performance analytics at a low cost and broad availability. 

This thesis primarily aims to understand the fundamental properties of a low-cost commercial-grade eye tracker. Such properties were analyzed and tested by applying various methods of eye movement classification. The efficacy of most eye movement classification methods has until now been proved only on data from research-grade eye trackers, which is why their application commercial-grade eye-tracking data is interesting.

Traditionally, methods for eye movement classification have relied on manually coded algorithms. This dependency introduces challenges for the algorithm implementer since every use case requires manually defined parameters, which vary significantly between implementations. Because of this, a machine learning model was also introduced, which could learn such parameters automatically based on a labeled dataset. The need for a dataset was solved by designing a data acquisition pipeline. This pipeline enabled a more efficient process of recording and labeling eye-tracking datasets. 

The author found that data from the eye-tracker used, although suffering from some deficiencies, was accurate and consistent enough to be used for most statistical inferences. Additionally, it was found that the data acquisition pipeline could streamline the process of recording and labeling data compared to manual methods. 
