\section{Eye Tracking Research}

As the reader might agree, vision is perhaps the most predominant sense of perception that we humans enjoy. From the raw signals of millions of optic nerve fibers, we can interpret vast amounts of information every second, and we know that a significant fraction of the cortex is involved in this process (\cite{klatzky2012}). As such, it is natural to believe that our eyes are a good metric when studying the internal cognitive functions of our brain. Cognitive psychologists have exploited this fact for over two centuries (\cite{eckstein2017}), with studies in fields ranging from education and marketing to neuroscience and artificial intelligence.

Some studies have undertaken ventures such as the possibility of inferring subject intention with eye movements. \cite{ballard1992}, for instance, wanted to investigate whether gaze patterns were an indicator that preceded actions. To do this, he conducted an experiment where subjects were to move a set of colored blocks to match the pattern of a given model, while a mobile eye tracker was set up to record their eye movements. He found that there was indeed a strong link between a subject's eye movements and their actions, with gaze following a clear pattern of checking out a block before picking it up, as well as the model before placing it down. Similarly, \cite{land1999} measured a subject's eye movements as they performed daily tasks, such as brewing tea or making a sandwich, again finding that the subject's eyes always revealed their intentions before acting.

% Marketing

Eye-tracking research in the field of education has been dominated by its use in investigative reading (\cite{knight2014}). Tracking the reader's saccadic movements between sentences and fixation times has provided profound insights into comprehension activity. \cite{vangog2009} applied this fact in a study into the attentional processes of expert versus novice readers and found clear contrasts in strategies employed for comprehension. Other studies (\cite{rayner1998}, \cite{rayner2008}) report that expert readers scan pages more quickly, continuously, and consistently, with an ability to concentrate attention on parts of the text that they believe to be critical. On the other hand, novice readers read linearly, displaying frustration and often giving up on the task. Studies such as these prove that eye-tracking is invaluable in assessing comprehension and other cognitive states.