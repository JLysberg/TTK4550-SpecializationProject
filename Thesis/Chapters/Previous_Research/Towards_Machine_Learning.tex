\newpage
\section{Towards Machine Learning} \label{sec:pw_TowardsMachineLearning}

Although this thesis aims to implement a machine learning model that can compete with the traditional algorithms mentioned above, it is not the first to attempt so. Both simple and very advanced machine learning methods have been implemented, usually with significant improvements in both autonomy and accuracy compared to traditional methods. Such algorithms are considered by many the new paradigm in eye movement event classification.

One breakthrough in this regard was made by \cite{zemblys2017}, where a Random Forest Classifier was utilized on a dataset recorded by a 1000Hz EyeLink 1000 eye tracker. Individual samples were labeled manually by an expert with nine years of eye-tracking experience into fixations, saccades, and \textit{post-saccading oscillations} (PSO). To accommodate different sampling frequencies, they augmented the dataset by resampling to 60, 120, 200, 250, 300, 500, and 1250Hz and adding multiple noise levels to each resample. The result was a model that beat traditional classification methods by large margins on the high-frequency dataset clean of noise. On the lowest-frequency dataset, the results were much worse. However, because of the manual features chosen for classification, many of which relied on sample-to-sample resolutions of less than 10ms, downsampled datasets had a significant disadvantage. The 50Hz dataset had sample-to-sample intervals of 20ms, for instance. Because of this, low-cost commercial eye trackers were not adequately represented.

A year later, many of the same authors produced a new model, presented in \cite{zemblys2018}. This model was fully end-to-end and capable of classifying fixations, saccades, and PSOs without the need for manually generated features. That was made possible by a deep neural network consisting of convolutional and recurrent layers. The model was trained on data recorded on a high-end tracker at 500Hz and manually labeled by experts. Results were in near-perfect agreement with manual coders. Recurrent neural networks also made it possible to train a generative network. This network was used to augment the training dataset by generating more of the less occurring labels, such as saccades. Doing so helped alleviate the problem of imbalanced frequency of occurrence of classes, which needs to be accounted for in this thesis.

% The most recent work on eye movement classification was done by \cite{fuhl2021}. They also developed a fully end-to-end model, with the improvement to \cite{zemblys2018} of additionally classifying smooth pursuits and noise. 