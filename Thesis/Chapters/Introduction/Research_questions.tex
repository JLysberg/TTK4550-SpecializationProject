\section{Research Questions} \label{sec:intro_research}

The choice of eye tracking hardware is entirely dependent on the use case, as eye trackers come with a range of properties, advantages and disadvantages. However, since I have stressed the aspect of commercial availability, that is naturally going to be the major deciding factor in the matter. The hardware of choice is described in detail in section \ref{sec:hwds_TobiiEyeTracker5}. \marginnote{Jeg vet at jeg er lite konsekvent på bruk av "we" eller "I", og innser at dette språket kanskje i utgangspunktet er en smule uformelt. Hva er fasit her? :)}The fact that we sacrifice some data accuracy and quality for availability reveals the first research question which this thesis aims to answer, that is; \textit{"How good is the data quality provided by a commercially available eye tracker, for the purposes of statistical inference by a machine learning model?"}

Since budgets limit research to data gathered from a single type of eye tracker, we sadly cannot directly compare one tracker from another. What we can do, however, is to compare potential limitations of our hardware with that of the publicly disclosed parameters of the more expensive eye trackers on the market. With this in mind, we will answer whether the commercially available eye tracker of choice will be sufficient for further research, or if manufacturers still need some years of development before performance analytics by eye tracking will be available to the general public.

Once hardware limitations have been addressed, we need a benchmark by which the data gathered might be assessed. As will be covered in section \ref{sec:pw_TraditionalClassificationMethods}, eye movement data streams have been used to classify eye movement events for decades, through manually coded algorithms and domain experience. What I aim to achieve is the development of a series of machine learning models that can contest these established practices, by answering; \textit{"Is eye movement data sufficient to train a machine learning model for eye event classification with a level of accuracy surpassing that of manually coded algorithms?"} 

Answering this will establish a niche within the field of performance analytics, facilitating further work and research towards more advanced models for inference on data enabled by eye tracking hardware.