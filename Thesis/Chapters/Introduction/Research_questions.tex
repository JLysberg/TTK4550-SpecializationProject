\section{Research Questions} \label{sec:intro_research}

The choice of eye-tracking hardware depends entirely on the use case, as eye-trackers come with a wide range of properties, advantages, and disadvantages. However, since the author has stressed the aspect of commercial availability, that is naturally going to be the primary deciding factor. The hardware of choice is described in detail in section \ref{sec:hwds_TobiiEyeTracker5}. The fact that we sacrifice some data accuracy and quality for availability reveals the first research question which this thesis aims to answer, that is; \textit{"How good is the data quality provided by a commercially available eye tracker, for the purposes of statistical inference by a machine learning model?"}

Since budgets limit research to data gathered from a single eye-tracker, we sadly cannot compare one tracker directly against another. What we can do, however, is to do an in-depth analysis of the data available and discuss possible alternatives where weaknesses are identified. With this in mind, we will answer whether the commercially available eye tracker of choice will be sufficient for further research or if manufacturers still need some years of development before performance analytics by eye-tracking will be open to the general public in the future.

Once hardware limitations have been addressed, we want to measure data quality by a benchmark. As will be covered in section \ref{sec:pr_TraditionalClassificationMethods}, researchers have used eye-movement data streams to classify eye movement events for decades through manually coded algorithms and domain experience. The author aims to develop a machine learning model that can contest these established practices by answering; \textit{"Is eye movement data sufficient to train a machine learning model for eye event classification with a level of accuracy surpassing that of manually coded algorithms?"} 

Finally, since the classification of pre-defined eye movement events is a problem that inherently requires a machine learning model trained by supervised learning, raw data by itself will likely not be enough to answer the above question adequately. For reasons that will be elaborated in section \ref{sec:bt_ArtificialIntelligence}, raw data needs to be labeled sample-by-sample with the eye events they represent. This problem brings us to the third and final research question for this thesis; \textit{Is it possible to develop an eye-tracking recording application that can control the environment to such a degree that sample labels can be inferred automatically?} 

% Answering this will establish a niche within the field of performance analytics, facilitating further work and research towards more advanced models for inference on data enabled by eye-tracking hardware.